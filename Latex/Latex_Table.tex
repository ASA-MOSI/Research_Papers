\documentclass{article}
\usepackage[utf8]{inputenc}
\usepackage{xcolor}%定义了一些颜色
\usepackage{colortbl,booktabs}%第二个包定义了几个*rule
\usepackage{tabularx}
\usepackage{tabu}
\usepackage{multicol}
\usepackage{multirow}
\usepackage{booktabs}
\usepackage{threeparttable}
\begin{document}
\begin{table}
\centering
\caption{Color Table}
\begin{tabular}
{>{\columncolor{blue}}rccccc}
\toprule[1pt]
\rowcolor[gray]{0.9}	&1 &2 	&3 	&4 	&5\\
\midrule
A	&\multicolumn{1}{>{\columncolor{green}[0pt][0pt]}c}{318.3} 	&327.8	&152.0 	&104.9 	&135.8\\
B 	&&\multicolumn{1}{>{\columncolor{red}[0pt][0pt]}r}{335.5} 	&137.7	&290.9 	&198.6\\
\bottomrule[1pt]
\end{tabular}
\end{table}

\begin{table}
\caption{Set Table Length}
\begin{tabular*}{12cm}{lll}
\hline
Start & End  & Character Block Name \\
\hline
3400  & 4DB5 & CJK Unified Ideographs Extension A \\
4E00  & 9FFF & CJK Unified Ideographs \\
\hline
\end{tabular*}
\end{table}

\begin{table}
\Large
\caption{Automatic line break}
\begin{center}
\begin{tabular}{|l|l|l|l| p{5cm}|}
\hline
Item & Name & Gender & Habit & Self-introduction \\ \hline
1 & Jimmy & Male & Badminton & Hi, everyone,my name is Jimmy. I come from Hamilton,
and it's my great honour to give this example. My topic is about how to use p{width} command \\ \hline
2 & Jimmy & Male & Badminton & Hi, everyone,my name is Jimmy. I come from Hamilton,
and it's my great honour to give this example. My topic is about how to use p{width} command \\
\hline
\end{tabular}
\end{center}
\end{table}

\begin{table}
\caption{Set Table Width x}
\begin{tabularx}{10cm}{llX}  % 10cm 減去前兩個欄位寬度後,剩下的通通給
\hline                      % 第三欄位使用,文字超出的部份會自動折行
Start & End  & Character Block Name  \\
\hline
3400  & 4DB5 & CJK Unified Ideographs Extension A \\
4E00  & 9FFF & CJK Unified Ideographs \\
\hline
\end{tabularx}
\end{table}

\begin{table}
\caption{Set the width of a column in the middle of the table}
\begin{tabularx}{12cm}{lXl}
\hline
Start & End  & Character Block Name \\
\hline
3400  & 4DB5 & CJK Unified Ideographs Extension A \\
4E00  & 9FFF & CJK Unified Ideographs \\
\hline
\end{tabularx}
\end{table}


\begin{table}[ht] %开始一个表格environment,表格的位置是h,here。
\caption{Change any column width} %显示表格的标题
\begin{tabular}{p{3.5cm}|p{2cm}|p{5cm}} %设置了每一列的宽度,强制转换。
\hline
\hline
Format & Extension & Description \\ %用&来分隔单元格的内容 \\表示进入下一行
\hline %画一个横线,下面的就都是一样了,这里一共有4行内容
Bitmap & .bmp & Bitmap images are recommended because they offer the most control over the exact image and colors.\\
\hline
Graphics Interchange Format (GIF) & .gif & Compressed image format used for Web pages. Animated GIFs are supported.\\
\hline
Joint Photographic Experts Group (JPEG) & .jpeg, .jpg & Compressed image format used for Web pages.\\
\hline
Portable Network Graphics (PNG) & .png & Compressed image format used for Web pages.\\
\hline
\hline
\end{tabular}
\end{table}

\begin{table}
\caption{tabu package}
\begin{center}
\begin{tabu} to 0.8\textwidth{X[c]|X[3,b]|X[2,l]|X[c]|X[3,m]|X[1,c]}
%0.8\textwidth   为设置表格宽度
%X[c]      表示这一列居中,所占比例为1,相当于X[1,c]
%X[3,c]   表示这一列居中,所占比例为3,这列的宽度是X[c]列的3倍
\hline
$i$  &$x_i$              &$n_i$      &$i$    &$x_i$               &$n_i$\\
\hline
1    &0.5$\sim$0.64       &1           &8    &1.48$\sim$1.62      &53\\
2    &0.64$\sim$0.78      &2           &9    &1.62$\sim$1.76      &25\\
3    &0.78$\sim$0.92      &9           &10   &1.76$\sim$1.90      &19\\
4    &0.92$\sim$1.06      &26          &11   &1.90$\sim$2.04      &16\\
5    &1.06$\sim$1.20      &37          &12   &2.04$\sim$2.18      &3\\
6    &1.20$\sim$1.34      &53          &13   &2.18$\sim$2.38      &1\\
7    &1.34$\sim$1.48      &56          &     &                    & \\
\hline
\end{tabu}
\end{center}
\end{table}

\newcommand{\tabincell}[2]{\begin{tabular}{@{}#1@{}}#2\end{tabular}}
\begin{table}[!t]
  \centering
  \scriptsize
  \caption{NOTATIONS}
  \label{tab:notations}
  \begin{tabular}{ll}
    \\[-2mm]
    \hline
    \hline\\[-2mm]
    {\bf \small Symbol}&\qquad {\bf\small Meaning}\\
    \hline
    \vspace{1mm}\\[-3mm]
    $P\!M_i$      &   \tabincell{l}{The $i\,th$ physical machine or host server in the data \\center, i = 1, 2, ?-}\\
    \vspace{1mm}
    $C\!M$          &  \tabincell{l}{ Vector of maximum disk size; $CM[i]$ stores the maximum\\ disk size of $PM_i$}\\
     \vspace{1mm}
    $B\!M$          &  \tabincell{l}{Vector of remaining disk size; $BM[i]$ stores the remaining\\ disk size of $PM_i$}\\
     \vspace{1mm}
    $S\!P(P\!M_i)$  &   \tabincell{l}{Selection preference of $PM_i$  }\\
     \vspace{1mm}
    $N\!ode_m$    &	\tabincell{l}{The m\,th node of the data center network. A node can be a  \\host server or a switch. m = 1, 2, ?-}\\
    \hline
    \hline
  \end{tabular}
\end{table}

\renewcommand{\arraystretch}{1.5} 
\begin{table}[tp]
 
  \centering
  \fontsize{6.5}{8}\selectfont
  \caption{Demographic Prediction performance comparison by three evaluation metrics.}
  \label{tab:performance_comparison}
    \begin{tabular}{|c|c|c|c|c|c|c|}
    \hline
    \multirow{2}{*}{Method}&
    \multicolumn{3}{c|}{C}&\multicolumn{3}{c|}{ D}\cr\cline{2-7}
    &Precision&Recall&F1-Measure&Precision&Recall&F1-Measure\cr
    \hline
    \hline
    A&0.7324&0.7388&0.7301&0.6371&0.6462&0.6568\cr\hline
   B&0.7321&0.7385&0.7323&0.6363&0.6462&0.6559\cr\hline
    C&0.7321&0.7222&0.7311&0.6243&0.6227&0.6570\cr\hline
    D&0.7654&0.7716&0.7699&0.6695&0.6684&0.6642\cr\hline
    E&0.7435&0.7317&0.7343&0.6386&0.6488&0.6435\cr\hline
    F&0.7667&0.7644&0.7646&0.6609&0.6687&0.6574\cr\hline
    G&{\bf 0.8189}&{\bf 0.8139}&{\bf 0.8146}&{\bf 0.6971}&{\bf 0.6904}&{\bf 0.6935}\cr
    \hline
    \end{tabular}
\end{table}

\renewcommand{\arraystretch}{1.5} %控制行高
\begin{table}[tp]
 
  \centering
  \fontsize{6.5}{8}\selectfont
  \begin{threeparttable}
  \caption{Demographic Prediction performance comparison by three evaluation metrics.}
  \label{performance_comparison}
    \begin{tabular}{ccccccc}
    \toprule
    \multirow{2}{*}{Method}&
    \multicolumn{3}{c}{ G}&\multicolumn{3}{c}{ G}\cr
    \cmidrule(lr){2-4} \cmidrule(lr){5-7}
    &Precision&Recall&F1-Measure&Precision&Recall&F1-Measure\cr
    \midrule
    kNN&0.7324&0.7388&0.7301&0.6371&0.6462&0.6568\cr
    F&0.7321&0.7385&0.7323&0.6363&0.6462&0.6559\cr
    E&0.7321&0.7222&0.7311&0.6243&0.6227&0.6570\cr
    D&0.7654&0.7716&0.7699&0.6695&0.6684&0.6642\cr
    C&0.7435&0.7317&0.7343&0.6386&0.6488&0.6435\cr
    B&0.7667&0.7644&0.7646&0.6609&0.6687&0.6574\cr
    A&{\bf 0.8189}&{\bf 0.8139}&{\bf 0.8146}&{\bf 0.6971}&{\bf 0.6904}&{\bf 0.6935}\cr
    \bottomrule
    \end{tabular}
    \end{threeparttable}
\end{table}

\renewcommand{\arraystretch}{1.2}
\begin{figure}[tbp]
\centering
\centerline{\bf (a). CDR samples}
\vspace{1.5mm}
\begin{tabular}{|l|c|c|}
\hline
\cline{1-1}
\underline{\textbf{record-id}} & \textbf{caller-id} & \textbf{callee-id} \\\hline 
1                                              & \#user-1           & \#user-2           \\\hline
2                                              & \#user-1           & \#user-4           \\\hline
3                                              & \#user-2           & \#user-1           \\\hline
4                                              & \#user3            & \#user-5\\\hline
5                                              & \#user1            & \#user-2\\\hline
\vdots                                             & \vdots           & \vdots\\\hline
\end{tabular}
 
\vspace{3mm}
\centering
\centerline{\bf (b). DTR samples}
\vspace{1.5mm}
\begin{tabular}{|l|c|c|c|}
\hline
\cline{1-1}
\textbf{record-id} & \textbf{user-id} & \textbf{online-time} & \textbf{offline-time} \\\hline
1    & \#user-1           & \#timestamp-1  & \#timestamp-2           \\\hline
2    & \#user-2           & \#timestamp-3  & \#timestamp-4           \\\hline
3    & \#user-2           & \#timestamp-5  & \#timestamp-6           \\\hline
4    & \#user3            & \#timestamp-7  & \#timestamp-8           \\\hline
\vdots & \vdots           & \vdots          &\vdots\\\hline
\end{tabular}
\vspace{1.5mm}
\caption{CDR (Call Detail Records) and DTR (Data Traffic Records) samples.}
\end{figure}
\end{document}
